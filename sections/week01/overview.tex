% !TeX root = ../../main.tex
\section{Some things to note and some Pre Reqs}
These notes will be similar to what is represented within the Computer Science Module within the University of Bristol however 
this is my way of somewhat not procrastinating so I am writing these notes 
and they will have my own touch to them. \\ \\ 
This and the rest of my projects can be found on my \textcolor{blue}{\href{https://github.com/mobeenmohammed}{GitHub}}
\\ \\ Things within the notes will be colour coded so that its easier visually to see what is what. 
\\ \\ For Example

\begin{theoremenv}[Example Theorem]
. $\forall \text{ } x \in \R$, $x^2 \ge 0$.
\end{theoremenv}


\subsection{Set Stuff and some Notation}

\begin{exampleenv}
. Some popular sets:
\begin{itemize}[nosep]
  \item Empty set: $\varnothing$
  \item Integers: $\Z = \{\cdots, -1, 0, 1, \cdots \}$
  \item Real Numbers $\R = (-\infty, \infty)$
\end{itemize}
\end{exampleenv}

\begin{definitionenv}
.   For sets $X$ and $Y$, $A$ and $B$, we have the following things:
\begin{itemize}[nosep]
    \item $X \subseteq Y \iff x \in X \implies x \in Y$
    \item $A \cap B = \{x \in X | \text{ } x \in A \land x \in B\}$
    \item $A \cup B = \{x \in X | \text{ } x \in A \lor x \in B\}$
    \item $A - B \iff A \setminus B =  \{x \in X | \text{ } x \in A \land x \notin B\}$
    \item $A^{\mathrm{c}} = \{x \in X | \text{ } x \notin A\}$
    \item The number of elements in a set $X$ is called the \textbf{cardinality} of a set denoted $|X|$
    \item For a set $X$ the set of all subsets of $X$ called the \textbf{power set} of $X$ is denoted $\mathcal{P}(X)$ is defined as $\mathcal{P}(X) = \{S | \text{ } S \subseteq X\}$
\end{itemize}
\end{definitionenv}

Semantics, Syntax and Computability are practically impossible without sets in general 
and also it helps us define functions.

\begin{definitionenv}[Predicates]
.   A \textbf{predicate} on a set $X$ is some subset $U \subseteq X$
\end{definitionenv}

We usually take a predicate to be some subset of our main set that has some useful things we can exploit. \\
A good example can be the set of \textbf{binary numbers}, $\mathbb{B} = \{0, 1\}$ being a predicate of $\Z$ or $\N$ depending on its interpretation.

\begin{definitionenv}[Cartesian Product]
.   For two sets $A$ and $B$ we define the \textbf{cartesian product} of them denoted $A \times B$ as the set of pairs 
    that contain one element from $A$ and the other from $B$ \\ \\
    $A \times B = \{(a,b) | \text{ } a \in A \land b \in B\}$
\end{definitionenv}

A \textbf{relation} from $A$ to $B$ is some subset of the cartesian product of the cartesian product of $A$ and $B$
\\ Now that I think about this, you're probably gonna see a lot of green during this.









